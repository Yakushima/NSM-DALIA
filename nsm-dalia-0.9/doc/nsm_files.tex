\documentclass[11pt]{article}
\usepackage{times}
\usepackage{pldoc}
\sloppy
\title{NSM INPUT FILES}
\makeindex

\begin{document}
% This LaTeX document was generated using the LaTeX backend of PlDoc,
% The SWI-Prolog documentation system
\maketitle
\hrule

\section{Overview}

An NSM input file is a text file, with some special tags which show to
NSM-DALIA what to do.

When NSM-DALIA parses such a file, it works in normal file-reading
mode (simply outputting each character found), until a tag is
encountered at the beginning of a new line. Tags begin with a "@"
character, followed by a letter. Tags are case-sensitive, so @p and @P
are different commands!

If you write an NSM input file with a word processor (instead of a
text editor), you will have to save your file in pure text format
(".txt"), ANSI-encoded (NOT unicode or UTF8. Support for such
encodings will be hopefully made available in later versions).

Here is a quick tour of the main tags implemented so far. Other will
be made available as the program is developed.

\begin{description}
    \item[@p] 
Following text is split into sentences (fullstop-separated), parsed
and transformed into NSM-PROLOG notation, until an @e tag is
found. Then, normal file-reading mode is resumed. You will find many
example of how this works in the "\file{sentences.txt}" file, in the demo
directory. See what happens when you run a pf("demo/\file{sentences.txt}")
command in NSM-DALIA: the sentences between a @p ... @e block are
replaced (in the output) by their NSM-PROLOG analysis, while all the
text outside such blocks is output as it is. Please note that the text
in a @p .. @e block is NOT in NSM format, but sentences must be
fullstop-separated, and indentation plays no role.
    \item[@P] 
Text in a @P ... @e block is considered to be in NSM- standard
format. See examples of this in the "\file{texts.txt}" file, in the "demo"
directory. The NSM standard format uses newline as sentence separator,
no punctuation, and indentation as a means of showing quotation. If
you want to split a long line, but you want the parser to consider the
new line as a continuation of the previous sentence, put a "/" or a
"\Sneg{}" at the end of the first part, and indentation and newline will be
not considered. Ex. in the following text, all indentations and
newlines are significative:
\end{description}

\begin{code}
this person thinks like this
    these people are not people like me
    these people are bad people
\end{code}

But you can also write it like this:

\begin{code}
this person thinks like this
    these people \
            are not people like me
    these people \
            are bad people
\end{code}

\begin{description}
    \item[@t] 
Text in a @t ... @e block is translated from current_language NSM into
curernt_l2 NSM. If current_language and current_l2 are the same, text
is not first parsed and then re-generated in the same language, but
simply output as it is. Text is in normal format, with fullstops
showing sentence boundaries.
    \item[@T] 
As before, but this time, text must be in NSM standard format.
    \item[@g] 
Text in a "@g ... end." block is read as NSM-PROLOG formulas, and
replaced by its equivalent in the target language. Note that this
block is ended by an "end." tag (with fullstop!), and not by the usual
"@e" tag. If generation fails for some reason, the formula is output
unchanged. You can see how this and the following tag work by making
NSM-DALIA parse the file "text-\file{gen.txt}", in the "demo" directory. Text
is in normal format.
    \item[@G] 
As before, but with text in NSM standard format. The end of the block
is given by and "end." command (with fullstop), instead of the usual
"@e" tag.
    \item[@@] 
This lets you write a "@" character at the beginning of a line
(writing a simple "@" in that position would start a tag-recognition
process).
\end{description}

\section{License}

This file is part of NSM-DALIA, an extensible parser and generator
for NSM grammars.

Copyright (C) 2009 Francesco Zamblera.

This program is free software: you can redistribute it and/or modify
it under the terms of the GNU General Public License as published by
the Free Software Foundation, either version 3 of the License, or
(at your option) any later version.

This program is distributed in the hope that it will be useful,
but WITHOUT ANY WARRANTY; without even the implied warranty of
MERCHANTABILITY or FITNESS FOR A PARTICULAR PURPOSE. See the
GNU General Public License for more details.

You should have received a copy of the GNU General Public License
along with this program. If not, see \url{http://www.gnu.org/licenses/}.
\printindex
\end{document}
